%%
%% defintns.tex - User defined commands for edengths.tex
%%
%% Copyright (C) 2010-2017 Mathew Topper <damm_horse@yahoo.co.uk>
%%
%%
%%   ABOUT
%%
%% This file contains the user defined commands for a Latex2e template which
%% corresponds to the regulations regarding layout of a thesis submitted within
%% the University of Edinburgh.

%%%%%%%%%%%%%%%%%%%%%%%%%%%%%%%%%%%%%%%%%%%%%%%%%%%%%%%%%%%%%%%%%%%%%%%%%%
%%%%%%%%%%             Define your commands here              %%%%%%%%%%%%
%%%%%%%%%%%%%%%%%%%%%%%%%%%%%%%%%%%%%%%%%%%%%%%%%%%%%%%%%%%%%%%%%%%%%%%%%%

%% New commands can be written using
%%    \newcommand{command}[inputs]{definition}.
%% In the definition the inputs are accessed with #1, #2, etc.
%%
%% If you want to override an existing command use \renewcommand instead
%% of \newcommand. \newcommand with give an error if command is already
%% defined.

%% If you are concerned that your command might override a default
%% you can use \providecommand which will ignore the new command if
%% a command of that name already exists.

%%%%% Some Example Maths Definitions (only use in maths mode)

\newcommand{\pdif}[2]{\frac{\partial #1}{\partial #2}}
%% ie \pdif{x}{t} would give partial x over t.

\newcommand{\dpdif}[2]{\dfrac{\partial #1}{\partial #2}}
%% inline partial derivative ie for $\dpdif{x}{t}$.
%% (\dfrac needs amsmath package)

\newcommand{\Ddif}[2]{\frac{D #1}{D #2}}
%% Material derivative

\newcommand{\spdif}[2]{\frac{\partial^{2} #1}{\partial #2^{2}}}
%% second partial derivative.

\newcommand{\altspdif}[3]{\frac{\partial^{2} #1}{\partial #2 \partial #3}} 
% mixed second partial i.e. \altspdif{x}{z}{t} = d2x / dzdt

\newcommand{\ndif}[2]{\frac{\mathrm{d} \, #1}{\mathrm{d} #2}}
%% ordinary differential.

\newcommand{\sndif}[2]{\frac{\mathrm{d} \,^{2} #1}{\mathrm{d} #2^{2}}}
%% ordinary second differential.
